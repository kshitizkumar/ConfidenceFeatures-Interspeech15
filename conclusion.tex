Confidence-scores have been used across speech applications in ROVER, arbitration, selecting high quality data for unsupervised model training, and key-word spotting etc. Confidence-scores are computed over a rich set of confidence-features in the speech recognition engine. While a number of downstream speech applications consume confidence scores, we haven't seen adequate focus on directly consuming confidence-features in those applications. We proposed that additionally consuming confidence-features can provide huge gains for confidence-related tasks and demonstrate that with respect to arbitration, where we obtained 31\% relative reduction in AUC metric. Furthermore, using confidence-features help decouple confidence-classifier and arbitration; this avoids a dependency on updating arbitration whenever we update confidence-classifier. We also demonstrate an application of confidence-scores in DNN speaker adaptation. Based on experiments we emphasize data with low confidence, this doubled WERR for DNN speaker adaptation on limited data scenarios.

%%\vspace{-3mm}
%\begin{equation}
%x(t) = s(f_\omega(t))
%\label{eq1}
%\end{equation}
%where \(f_\omega(t)\) is a special warping function
%\begin{equation}
%f_\omega(t)=\frac{1}{2\pi j}\oint_C \frac{\nu^{-1k}d\nu}
%{(1-\beta\nu^{-1})(\nu^{-1}-\beta)}
%\label{eq2}
%\end{equation}
%A residue theorem states that
%\begin{equation}
%\oint_C F(z)dz=2 \pi j \sum_k Res[F(z),p_k]
%\label{eq3}
%\end{equation}
%Applying (\ref{eq3}) to (\ref{eq1}),
%it is straightforward to see that
%\begin{equation}
%1 + 1 = \pi
%\label{eq4}
%\end{equation}
%
%\begin{figure}[t]
%\centerline{\epsfig{figure=figure,width=80mm}}
%\caption{{\it Schematic diagram of speech production.}}
%\label{spprod}
%\end{figure}










%
%\begin{algorithm}{IfForWhile}{
%\label{algo:ifforwhile}
%\qcomment{demonstrates control structures}}
%\qfor $i \qlet 1$ \qto $n$ \\
%\qdo $x_{i} \qlet x_{i}^{2}$; \\
%$y_{i} \qlet x_{i} - y_{i}$ \qrof\\
%\qif $A = B$ \label{line:ifAisB}\\
%\qthen do whatever is necessary if $A$ equals $B$\\
%\qelse do something else\\
%and wait for better times \qfi\\
%\qwhile $Q \neq \emptyset$ \\
%\qdo let $q$ be the first element of $Q$ and remove it from $Q$\\
%do something with $q$ \qend \\
%\qrepeat \\
%do something really weird
%\quntil you get sufficiently tired of it\\
%\qreturn $42$
%\end{algorithm}


%\begin{algorithm}
%\SetAlgoLined
%\KwData{this text}
%\KwResult{how to write algorithm with \LaTeX2e }
%initialization\;
%\While{not at end of this document}{
%read current\;
%\eIf{understand}{
%go to next section\;
%current section becomes this one\;
%}{
%go back to the beginning of current section\;
%}
%}
%\caption{How to write algorithms}
%\end{algorithm}
