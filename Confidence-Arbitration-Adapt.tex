\documentclass[a4paper]{article}
\usepackage{INTERSPEECH2014,amssymb,amsmath,epsfig, epstopdf}
%\usepackage[named]{algo}
%\usepackage[numbered]{algo}
\setcounter{page}{1}
\sloppy     % better line breaks
\ninept 
%SM below a registered trademark definition
\def\reg{{\rm\ooalign{\hfil
     \raise.07ex\hbox{\scriptsize R}\hfil\crcr\mathhexbox20D}}}

%% \newcommand{\reg}{\textsuperscript{\textcircled{\textsc r}}}
\title{Confidence-Features and Confidence-Scores for ASR applications in Arbitration and DNN Speaker Adaptation}

%%%%%%%%%%%%%%%%%%%%%%%%%%%%%%%%%%%%%%%%%%%%%%%%%%%%%%%%%%%%%%%%%%%%%%%%%%
%% If multiple authors, uncomment and edit the lines shown below.       %%
%% Note that each line must be emphasized {\em } by itself.             %%
%% (by Stephen Martucci, author of spconf.sty).                         %%
%%%%%%%%%%%%%%%%%%%%%%%%%%%%%%%%%%%%%%%%%%%%%%%%%%%%%%%%%%%%%%%%%%%%%%%%%%
%\makeatletter
%\def\name#1{\gdef\@name{#1\\}}
%\makeatother
%\name{{\em Firstname1 Lastname1, Firstname2 Lastname2, Firstname3 Lastname3,}\\
%      {\em Firstname4 Lastname4, Firstname5 Lastname5, Firstname6 Lastname6,
%      Firstname7 Lastname7}}
%%%%%%%%%%%%%%% End of required multiple authors changes %%%%%%%%%%%%%%%%%

\makeatletter
\def\name#1{\gdef\@name{#1\\}}
\makeatother \name{{\em Kshitiz Kumar, Ziad Al Bawab, Yong Zhao, Chaojun Liu, Yifan Gong}}

\address{Microsoft Corporation, Redmond, WA \\
{\small \tt \{kshitiz.kumar, Ziadal, yonzhao, Chaojunl, yifan.gong\}@microsoft.com}}

%
\begin{document}
\maketitle

\begin{abstract}
Speech recognition confidence-scores quantitatively represent the correctness of
decoded utterances in a [0,1] range. Confidences have primarily been used to filter out recognitions
with scores below a threshold. They have also been used in other speech applications in \emph{e.g.} Arbitration, ROVER
and high-quality data selection for model training etc. Confidence-scores are computed from a rich set of confidence-features in the speech recognition engine. While many speech applications consume confidence scores, we haven't seen adequate focus on directly consuming confidence-features in applications. In this work we build a thesis that additionally consuming confidence-features can provide big gains across confidence-related tasks. We demonstrate this for Arbitration application, where we obtain 31\% relative reduction in arbitration metric. We additionally demonstrate a novel application of confidence-scores in deep-neural-network (DNN) adaptation, where we can double the relative reduction in word-error-rate (WER) for speaker adaptation on limited data.
\end{abstract}
\noindent{\bf Index Terms}: Speech recognition, Confidence scores, Confidence predictors, Classifier, MLP

%
\section{Introduction}\label{Sec:Intro}
Automatic speech recognition (ASR) has seen the strongest wave of deployment and usage across devices and services in recent years. Confidence-scores are integral to ASR, we obtain these scores from a confidence-classifier trained over a set of confidence-features to maximally discriminate between correct and incorrect recognitions. We refer \cite{Posen1} for an introduction to our confidence classifier framework. The Confidence-scores that lie in a [0,1] range, we desire higher scores for correct recognitions, and lower for, (a) incorrect recognitions from in-grammar (IG) and, (b) any recognition from out-of-grammar (OOG) utterances. These scores are typically evaluated for individual words as well as the utterance. Historically confidences were used for ASR-enabled devices that are always in an active (continuously) listening mode in an application-constrained grammar. There potential recognitions from side-speech, background noise etc. can trigger unexpected system response. Therefore, confidence-scores were used to contain recognitions from OOG utterances from being recognized as IG utterances. We refer \cite{CMsurvey_Jiang_SpeechCommunication06, NBest_Wessel_Eurospeech99, MaximumEntropyConfidence_White_ICASSP07,Blatz04confidenceestimation,Rose_UV_1995,Mathan_Rejection_1991,Sukkar_UV_1996} for a survey of confidence techniques and specifically \cite{WordLattice_Kemp_Eurospeech97, MaximumEntropyConfidence_White_ICASSP07, Wessel01ney,Chase_wordand,Weintraub97}
for features and the classifiers used.

Confidence-scores have also been used in other ASR applications {e.g.}, (a) Arbitration where we select the best between client and service recognition results, (b) ROVER where we perform multi-system combination, (c) selecting high quality data for unsupervised model training, (d) key-word spotting tasks, (e) confidence-normalization etc. While many of the downstream ASR applications consume confidence-scores we have seen limited attempts on additionally consuming confidence-features. In this work we present our individual confidence-features, and, highlight the diverse information they encapsulate. We specifically demonstrate the richness of these features for Arbitration application where we present significant gains in arbitration metric. 

In addition to emphasizing the importance of confidence-features, we also present a novel application of confidence-scores by embedding them in the DNN speaker adaptation framework. We have already established significant gains with our baseline speaker adaptation with limited utterances on the speaker-independent DNN model. There we embed confidence-scores into the DNN update and obtain additional gains over our best baseline adaptation.

We have seen a related framework in \cite{Nuance_Arbitration}.

Rest of this work is organized in the following. We provide a background to our confidence-features and confidence-scores in Sec.~\ref{Sec:CC-Background}. We discuss an application of these features to Arbitration in Sec.~\ref{Sec:Arbitration}. We present a new novel application of confidence-scores to further improve our current best DNN adaptation in Sec.~\ref{Sec:Adaptation}. We provide new scope and application of confidence-features and scores in Sec.~\ref{Sec:Discussion} and Sec.~\ref{Sec:Conclusion} concludes our study.

% The classifiers are trained from a specified set of acoustic model (AM), grammar and speech data.

\section{Background on Confidence-Features and Confidence-Scores}\label{Sec:CC-Background}
We discussed the significance of confidence-scores in Section~\ref{Sec:Intro} where we mentioned that
confidence-classifier makes an inference on the correctness of recognition events. This is thus a binary
classification problem \cite{Bishop} with the 2-classes in (1) correct recognitions, (2) all incorrect
recognitions that includes misrecognitions over IG utterances as well any recognition
from OOG utterances. The classifier is trained from a rich set of confidence-features that we obtain from speech decoding.
A few of our prominent confidence-features are:
\begin{enumerate}
  \item acoustic-model features - we aggregate per-frame acoustic score over a word or an utterance. We also compute scores from acoustic arc transitions. These scores are typically normalized for duration.
  \item language-model features - these include fanout and perplexity features.
  \item noise and silence-model features - we compute features from noise and silence models.
  \item 2nd-order features - we compute word-confidence-weighted average of acoustic features in a phrase, see \cite{Posen1}.
  \item duration features - we compute word-duration and number of words in a phrase etc.
  \item senone count - count of active senones during decoding.
  \item confusibility - this indicates confusibility of the best hypothesis
  \item log-spectra-derived features - we may derive posterior features from speech log-spectra.
\end{enumerate}
Our features are appropriately normalized to be robust to speech with different duration and intensity. We refer to \cite{Posen1} for additional details to our confidence-classifier architecture and related features. Though a number of speech applications consume confidence-scores, we have seen limited attempts on directly consuming the rich set of confidence-features in those applications. Confidence-scores are obtained from a confidence-classifier trained over a particular collection of training data and grammar, from which we obtain positive tokens from successfully recognized utterances and negatives from incorrect recognitions. 

Thus confidence-scores are optimized for the purposes of classifying correct and incorrect recognitions. The optimization criterion and corresponding needs can be different for downstream speech applications that currently consume confidence-scores. In this work we motivate a widespread use of the rich set of confidence-features. These features are typically 15-20 for a word or for an utterance, so additional memory required to store these features is minimal. These confidence-features are already computed for the purpose of confidence-score so additional work required for extracting confidence-features is almost none. Furthermore, these confidence-features will also need to be communicated along with ASR result to the downstream consumer - considering a typical speech segment of 4 secs. at 8~kB/sec for a total of 32kB, confidence-features just add 80 Bytes to the communication footprint, thus less than 0.2\% to speech footprint.



\section{Rich Confidence-Features for Arbitration}\label{Sec:Arbitration}
Arbitration is an application where we select the best among multiple simultaneous ASR results. We explain our arbitration framework for personal assistant experience on smart devices in Fig.~\ref{Fig:Arbitration}. We decode an utterance simultaneously for both client and service engines. Client engine is designed to work with traditional client scenarios like \emph{call, digit dialing, text, open applications etc.}, service engine works better for rest of the speech scenarios including \emph{voice-search, weather etc.}. By design client and service cater seamlessly to all speech scenarios and contain language-model (LM) and acoustic-model (AM) optimized for respective tasks. Though we have distinct engines for client and service speech scenarios, they work together in a unified way that is indistinct for the user as we obviously don't expect user to provide us inputs on his scenario being one of client or service. Arbitration is the key speech application that provides a unified experience by selecting the best among the client and service results.  In Fig.~\ref{Fig:Arbitration} both client and service listen to speech from potentially all scenarios and produce respective recognition results under the constraint of their respective engine, AM and LM. These results are communicated to arbitration where it selects the best between the two results. Arbitration sends the results back to client where a decision unit at client will typically provide the arbitrated result to the user on their smart devices.

There can be a few scenarios where the decision unit can simply choose the client recognition \emph{e.g.}, (a) if client confidence-scores are higher than a present threshold; client can simply choose client ASR result if it's very confident, this avoids latency incurred in hearing back from service and arbitration, (b) in absence of connection to service, then user can still use client side speech applications. %Microsoft or other $3^{rd}$-party applications.

% This framework also allows us to cater to scenarios where a user says: ``text Alex, I will be late". There client engine has access to contacts and can recognize the person name ``Alex" and produce a result like ``text Alex ...", then ``..." will be filled in by recognition from service. Client AM contains garbage paths that greatly help absorb utterances are intended for service only and thus are out-of-grammar (OOG) for client.

%\subsection{Arbitration Classifier and Baseline Features}\label{Sec:ArbitrationCC}
%Brief description of current arbitration framework, and current useful features

\begin{figure}[h]
\centering
{\includegraphics[width=0.48\textwidth]{Arbitration}}
\caption{\it Arbitration for client and service ASR results. We additionally feed confidence-features from both client and server to arbitration.}
\label{Fig:Arbitration}
\end{figure}

\subsection{Incorporating Confidence-Features in Arbitration}
% We described our arbitration framework and the baseline features in \ref{Sec:ArbitrationCC}, there we mentioned a number of features that we found very relevant for arbitration. 

In this section, we build a thesis on consuming the rich set of confidence-features in arbitration. 
Our baseline arbitration is trained from a number of carefully designed features, still the confidence-features provide detailed and complementary information in noise, silence, acoustic and language-model scores, and is expected to be useful for arbitration. Confidence-scores from both client and service are currently used by arbitration, these scores provide a good gist of confidence-features but we can benefit a great deal with directly consuming confidence-features by (a) using much more gradual information in terms of 20-dim confidence-features versus a single confidence-score in arbitration, (b) confidence-scores are designed to optimize the performance of confidence-classifier which is clearly different from arbitration, so retraining with confidence-features helps, (c) arbitration and confidence-classifier may be trained over different datasets, so the information encapsulated by confidence-score may not generalize to dataset relevant for arbitration, (d) confidence-scores are language-specific as they may have been individually trained across a set of \emph{AM, LM, languages, dataset}, in contrast, we have noted that the various inherent normalizations in confidence-features make them robust across locals, so consuming confidence-features can allow us to build an arbitration classifier from one local that can provide good performance for other unseen locals; this can be specially useful when we bootstrap arbitration for a local under limited data scenarios, (e) using confidence-score in arbitration creates a dependency for arbitration on confidences, any update to confidence-classifiers potentially requires retraining arbitration; we can completely alleviate this issue if we consume confidence-features instead of confidence-score in arbitration, this lets us independently update confidence-classifier without impacting arbitration.

We demonstrate our approach that uses the rich confidence-features for arbitration in Fig.~\ref{Fig:Arbitration}. We build the infrastructure required to extract confidence-features from both client and service engine, and communicate them to arbitration, see Fig.~\ref{Fig:Arbitration}. As expected we require little additional work in communicating service confidence-features to arbitration. For client, we additionally send about 30 Bytes-per-second of data, this is less than 0.2\% relative increase to our payload from client to service. Depending on application and need, arbitration can be retrained and deployed with confidence-features from (a) just client, (b) just service, (c) both client and service. 

% 3 second, 20 float features = 20 * 4 = 80 bytes = 30bytes/sec
% speech data = 8000 bytes

\subsection{Arbitration Experiments and Results}\label{Sec:ArbitrationResults}
We present and analyze results with using confidence-features in arbitration.
Our arbitration module was trained from over 35k speech utterances. Testing was done on over 25k utterances. We decoded these utterances against both client and service engines with their respective  AM and LM, and obtained corresponding recognition results. We had ground-truth transcriptions for these utterances and created their classification targets in terms of client or service based on what provided a lower word-error rate. Our baseline arbitration features are 20-dim, that include duration, confidence-score and few semantic features etc. We additionally obtained 20-dim confidence-features from each of client and service. We followed the existing framework for training arbitration that uses a boosted-decision-tree for classification.

% Note that ideally clients will have distinct personalized grammars with their own contact names, application names etc. 

% where we additionally included 20 confidence-features. 

%For our purposes we simulated client grammar with each over 250 names in contacts and used these grammars in decoding. 

We demonstrate the value in client confidence-features in Fig.~\ref{Fig:PhrasePreds-Hist}. There we plot probability-distribution-function for a few features for arbitration task. There ``correct" refers to cases where client wins, and ``InCorrect" refers to service wins, ``Confidence-Score" indicates usual client confidence-score. We visually see that some of the features better separate the 2 classes than confidence-score.

\begin{figure}[h]
\centering
{\includegraphics[width=0.45\textwidth]{PhrasePreds-Hist.eps}}
\caption{\it Probability-distribution of representative confidence-features for arbitration.}
\label{Fig:PhrasePreds-Hist}
\end{figure}

Next we provide receiver-operating-curve (ROC) for baseline and with including client confidence-features in Fig.~\ref{Fig:Baseline-ClientPred-ROC}, where we note a strongly better ROC curve throughout the range of curve. In this arbitration task, ``False Positive" (FP) indicates incorrect wins from client and ``True Positive" (TP) indicates correct wins from service. Specifically at FP of 0.1, we can improve TP from 0.81 to 0.86, for a 26\% relative reduction in (1-TP). We note correct area-under-the-curve (AUC) metrics in Table~\ref{tab:AUC_ROC}, where including client confidence-features improved AUC from 0.927 for baseline to 0.946, additionally including server confidence-features improved AUC to 0.95 for a 31.5\% relative reduction in (1-AUC) metric. Our arbitration-classifier ranks all of it's features in the order of importance. As expected confidence-features appear prominently among the top features, with 7 of the top-10 overall features being confidence-features. This further demonstrates that the proposed features outperform and add value to the baseline features.

\begin{figure}[h]
\centering
{\includegraphics[width=0.4\textwidth]{Baseline-ClientPred-ROC}}
\caption{\it Receiver operating curve (ROC) for arbitration.}
\label{Fig:Baseline-ClientPred-ROC}
\end{figure}

\begin{table}
\begin{center}
\begin{small}
\caption{Area-under-the-curve (AUC) for ROC chart} \label{tab:AUC_ROC}
\begin{tabular}{|l|c|c|}
\hline
Method & AUC & relative reduction in \\
& & (1-AUC) [\%]\\
\hline
Baseline Features & 0.927 & - \\
\hline
+ Client Confidence-Features & 0.946 & 26.0\\
\hline
+ Client and Service & &\\
~~Confidence-Features & 0.950 & 31.5\\
\hline
\end{tabular}
\end{small}
\end{center}
\end{table}


\section{Confidence-Scores in DNN Speaker Adaptation Framework}\label{Sec:Adaptation}

We have noted that speaker adaptation is an area where we have significant avenue for further improvements.
In this section, we propose a novel application of confidence-score for DNN speaker adaptation work for limited and large adaptation data scenarios. We have noted that confidence-scores imply the correctness of recognition results. In the context of speaker-adaptation confidences also indicate a degree of match between the speaker-dependent data and speaker-independent model. Thus we can leverage confidence-score in the DNN speaker adaptation optimization by disproportionately weighting data across confidence-scores. Correspondingly we created a new recipe where we first bucket all of adaptation data into 3 buckets for low, medium and high confidence-scores. Our goal is to alter the optimization metric by including confidence-scores. Corresponding to the 3 confidence-categories we can weight the data samples from those categories according to specified values for those categories.


We know that confidence can indicate a great deal about the quality of utterances that we use in adaptation but none of the current adaptation recipes include confidence, specifically:
For correct hypothesis	For incorrect hypothesis, (a) low confidence data is a poor match to model and may benefit with higher weight on the data, (b) low confidence results are likely to be incorrect, so we should deemphasize these utterances, (c) high confidence data is already a good match to model, so there is less to learn from this data, (d) high confidence results are likely to be correct, so we should emphasize these utterances.

For 50 utts we can improve WERR from 11.6\% to 14.1\% for supervised adaptation.

Approach. New confidence recipe 
Split an utterance alignment into individual words
Group the words bases on their confidences in low, medium and high confidence buckets
We can disproportionately duplicate low, medium and high confidence words
Testing remains unchanged, where we test on the entire utterance without splitting them into word-level
Applied this recipe to training and testing on 6 speaker adaptation data on SMD task
Based on experiments selected a recipe that simply duplicates the low and medium confidence buckets while retaining the high confidence bucket. Noting results for unsupervised and supervised adaptation in below

\subsection{DNN adaptation experiment}
Training vs test Dataset
Server task
Large LM
Any difference of above 1\% relative is significant.



\begin{table}
\begin{center}
\begin{small}
\caption{WER for Supervised adaptation. Baseline WER is 19.9\%.} \label{tab:mean-FA-diff}
\begin{tabular}{|c|c|c|c|c|c|}
\hline
Nutts &  \multicolumn{2}{|c|}{Best Adaptation} & \multicolumn{2}{|c|}{+Include} \\
&  \multicolumn{2}{|c|}{ }  & \multicolumn{2}{|c|}{Confidence-Score} \\
\hline
 &   WER & WERR & WER & WERR\\
\hline
20 &  19.6  & 1.5  & 18.9 & 5.0\\
50 &  17.6  & 11.6  & 17.1 & 14.1\\
100 &  16.7  & 16.1  & 16.4 & 17.6\\
\hline
\end{tabular}
\end{small}
\end{center}
\end{table}

\begin{table}
\begin{center}
\begin{small}
\caption{WER for Unsupervised adaptation. Baseline WER is 19.9\%.} \label{tab:mean-FA-diff}
\begin{tabular}{|c|c|c|c|c|c|}
\hline
Nutts &  \multicolumn{2}{|c|}{Best Adaptation} & \multicolumn{2}{|c|}{+Include} \\
&  \multicolumn{2}{|c|}{ }  & \multicolumn{2}{|c|}{Confidence-Score} \\
\hline
 &   WER & WERR & WER & WERR\\
\hline
20 &  20.2  & -1.4  & 19.9 & 0.2\\
50 &  19.0  & 4.9  & 18.2 & 8.5\\
100 &  18.0  & 9.8  & 17.7 & 11.3\\
\hline
\end{tabular}
\end{small}
\end{center}
\end{table}


\section{Discussion}\label{Sec:Discussion}
We demonstrated novel applications of confidence-features and confidence-scores in this work, where we presented strong gains for arbitration and DNN speaker adaptation. We also foresee an application of confidence-features to in ROVER for system combination. Confidence-score is one of the strongest features for  ROVER system but these confidence-scores can be in different range across the individual systems, thus they required as additional step of normalizing the confidence-scores. We have noted that Confidence-features generalize across acoustic models and thus avoid the requirement of normalization step. Confidence-scores are also used in model recommendation for identical LM, where we recommend one of the many AMs that may work best in particular scenarios. There too we can leverage confidence-features for additional gain.


\section{Conclusions}\label{Sec:Conclusion}
Confidence-scores have been used across speech applications in ROVER, arbitration, selecting high quality data for unsupervised model training, and key-word spotting etc. Confidence-scores are computed over a rich set of confidence-features in the speech recognition engine. While a number of downstream speech applications consume confidence scores, we haven't seen adequate focus on directly consuming confidence-features in those applications. We proposed that additionally consuming confidence-features can provide huge gains for confidence-related tasks and demonstrate that with respect to arbitration, where we obtained 31\% relative reduction in AUC metric. Furthermore, using confidence-features help decouple confidence-classifier and arbitration; this avoids a dependency on updating arbitration whenever we update confidence-classifier. We also demonstrate an application of confidence-scores in DNN speaker adaptation. Based on experiments we emphasize data with low confidence, this doubled WERR for DNN speaker adaptation on limited data scenarios.

%%\vspace{-3mm}
%\begin{equation}
%x(t) = s(f_\omega(t))
%\label{eq1}
%\end{equation}
%where \(f_\omega(t)\) is a special warping function
%\begin{equation}
%f_\omega(t)=\frac{1}{2\pi j}\oint_C \frac{\nu^{-1k}d\nu}
%{(1-\beta\nu^{-1})(\nu^{-1}-\beta)}
%\label{eq2}
%\end{equation}
%A residue theorem states that
%\begin{equation}
%\oint_C F(z)dz=2 \pi j \sum_k Res[F(z),p_k]
%\label{eq3}
%\end{equation}
%Applying (\ref{eq3}) to (\ref{eq1}),
%it is straightforward to see that
%\begin{equation}
%1 + 1 = \pi
%\label{eq4}
%\end{equation}
%
%\begin{figure}[t]
%\centerline{\epsfig{figure=figure,width=80mm}}
%\caption{{\it Schematic diagram of speech production.}}
%\label{spprod}
%\end{figure}










%
%\begin{algorithm}{IfForWhile}{
%\label{algo:ifforwhile}
%\qcomment{demonstrates control structures}}
%\qfor $i \qlet 1$ \qto $n$ \\
%\qdo $x_{i} \qlet x_{i}^{2}$; \\
%$y_{i} \qlet x_{i} - y_{i}$ \qrof\\
%\qif $A = B$ \label{line:ifAisB}\\
%\qthen do whatever is necessary if $A$ equals $B$\\
%\qelse do something else\\
%and wait for better times \qfi\\
%\qwhile $Q \neq \emptyset$ \\
%\qdo let $q$ be the first element of $Q$ and remove it from $Q$\\
%do something with $q$ \qend \\
%\qrepeat \\
%do something really weird
%\quntil you get sufficiently tired of it\\
%\qreturn $42$
%\end{algorithm}


%\begin{algorithm}
%\SetAlgoLined
%\KwData{this text}
%\KwResult{how to write algorithm with \LaTeX2e }
%initialization\;
%\While{not at end of this document}{
%read current\;
%\eIf{understand}{
%go to next section\;
%current section becomes this one\;
%}{
%go back to the beginning of current section\;
%}
%}
%\caption{How to write algorithms}
%\end{algorithm}


%\newpage
%
%\eightpt
\bibliographystyle{IEEEbib}
\bibliography{strings}
\end{document}

