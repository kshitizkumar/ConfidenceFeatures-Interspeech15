
We have noted that speaker adaptation is an area where we have significant avenue for further improvements.
In this section, we propose a novel application of confidence-score for DNN speaker adaptation work for limited and large adaptation data scenarios. We have noted that confidence-scores imply the correctness of recognition results. In the context of speaker-adaptation confidences also indicate a degree of match between the speaker-dependent data and speaker-independent model. Thus we can leverage confidence-score in the DNN speaker adaptation optimization by disproportionately weighting data across confidence-scores. Correspondingly we created a new recipe where we first bucket all of adaptation data into 3 buckets for low, medium and high confidence-scores. Our goal is to alter the optimization metric by including confidence-scores. Corresponding to the 3 confidence-categories we can weight the data samples from those categories according to specified values for those categories.


We know that confidence can indicate a great deal about the quality of utterances that we use in adaptation but none of the current adaptation recipes include confidence, specifically:
For correct hypothesis	For incorrect hypothesis, (a) low confidence data is a poor match to model and may benefit with higher weight on the data, (b) low confidence results are likely to be incorrect, so we should deemphasize these utterances, (c) high confidence data is already a good match to model, so there is less to learn from this data, (d) high confidence results are likely to be correct, so we should emphasize these utterances.

For 50 utts we can improve WERR from 11.6\% to 14.1\% for supervised adaptation.

Approach. New confidence recipe 
Split an utterance alignment into individual words
Group the words bases on their confidences in low, medium and high confidence buckets
We can disproportionately duplicate low, medium and high confidence words
Testing remains unchanged, where we test on the entire utterance without splitting them into word-level
Applied this recipe to training and testing on 6 speaker adaptation data on SMD task
Based on experiments selected a recipe that simply duplicates the low and medium confidence buckets while retaining the high confidence bucket. Noting results for unsupervised and supervised adaptation in below

\subsection{DNN adaptation experiment}
Training vs test Dataset
Server task
Large LM
Any difference of above 1\% relative is significant.



\begin{table}
\begin{center}
\begin{small}
\caption{WER for Supervised adaptation. Baseline WER is 19.9\%.} \label{tab:mean-FA-diff}
\begin{tabular}{|c|c|c|c|c|c|}
\hline
Nutts &  \multicolumn{2}{|c|}{Best Adaptation} & \multicolumn{2}{|c|}{+Include} \\
&  \multicolumn{2}{|c|}{ }  & \multicolumn{2}{|c|}{Confidence-Score} \\
\hline
 &   WER & WERR & WER & WERR\\
\hline
20 &  19.6  & 1.5  & 18.9 & 5.0\\
50 &  17.6  & 11.6  & 17.1 & 14.1\\
100 &  16.7  & 16.1  & 16.4 & 17.6\\
\hline
\end{tabular}
\end{small}
\end{center}
\end{table}

\begin{table}
\begin{center}
\begin{small}
\caption{WER for Unsupervised adaptation. Baseline WER is 19.9\%.} \label{tab:mean-FA-diff}
\begin{tabular}{|c|c|c|c|c|c|}
\hline
Nutts &  \multicolumn{2}{|c|}{Best Adaptation} & \multicolumn{2}{|c|}{+Include} \\
&  \multicolumn{2}{|c|}{ }  & \multicolumn{2}{|c|}{Confidence-Score} \\
\hline
 &   WER & WERR & WER & WERR\\
\hline
20 &  20.2  & -1.4  & 19.9 & 0.2\\
50 &  19.0  & 4.9  & 18.2 & 8.5\\
100 &  18.0  & 9.8  & 17.7 & 11.3\\
\hline
\end{tabular}
\end{small}
\end{center}
\end{table}
