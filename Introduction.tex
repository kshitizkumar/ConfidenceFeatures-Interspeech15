Automatic speech recognition (ASR) has seen the strongest wave of deployment and usage across devices and services in recent years. Confidence-scores are integral to ASR, we obtain these scores from a confidence-classifier trained over a set of confidence-features to maximally discriminate between correct and incorrect recognitions. We refer \cite{Posen1} for an introduction to our confidence classifier framework. The Confidence-scores that lie in a [0,1] range, we desire higher scores for correct recognitions, and lower for, (a) incorrect recognitions from in-grammar (IG) and, (b) any recognition from out-of-grammar (OOG) utterances. These scores are typically evaluated for individual words as well as the utterance. Historically confidences were used for ASR-enabled devices that are always in an active (continuously) listening mode in an application-constrained grammar. There potential recognitions from side-speech, background noise etc. can trigger unexpected system response. Therefore, confidence-scores were used to contain recognitions from OOG utterances from being recognized as IG utterances. We refer \cite{CMsurvey_Jiang_SpeechCommunication06, NBest_Wessel_Eurospeech99, MaximumEntropyConfidence_White_ICASSP07,Blatz04confidenceestimation,Rose_UV_1995,Mathan_Rejection_1991,Sukkar_UV_1996} for a survey of confidence techniques and specifically \cite{WordLattice_Kemp_Eurospeech97, MaximumEntropyConfidence_White_ICASSP07, Wessel01ney,Chase_wordand,Weintraub97}
for features and the classifiers used.

Confidence-scores have also been used in other ASR applications {e.g.}, (a) Arbitration where we select the best between client and service recognition results, (b) ROVER where we perform multi-system combination, (c) selecting high quality data for unsupervised model training, (d) key-word spotting tasks, (e) confidence-normalization etc. While many of the downstream ASR applications consume confidence-scores we have seen limited attempts on additionally consuming confidence-features. In this work we present our individual confidence-features, and, highlight the diverse information they encapsulate. We specifically demonstrate the richness of these features for Arbitration application where we present significant gains in arbitration metric. 

In addition to emphasizing the importance of confidence-features, we also present a novel application of confidence-scores by embedding them in the DNN speaker adaptation framework. We have already established significant gains with our baseline speaker adaptation with limited utterances on the speaker-independent DNN model. There we embed confidence-scores into the DNN update and obtain additional gains over our best baseline adaptation.

We have seen a related framework in \cite{Nuance_Arbitration}.

Rest of this work is organized in the following. We provide a background to our confidence-features and confidence-scores in Sec.~\ref{Sec:CC-Background}. We discuss an application of these features to Arbitration in Sec.~\ref{Sec:Arbitration}. We present a new novel application of confidence-scores to further improve our current best DNN adaptation in Sec.~\ref{Sec:Adaptation}. We provide new scope and application of confidence-features and scores in Sec.~\ref{Sec:Discussion} and Sec.~\ref{Sec:Conclusion} concludes our study.

% The classifiers are trained from a specified set of acoustic model (AM), grammar and speech data.